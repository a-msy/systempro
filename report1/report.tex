%概要、課題について(課題ごとにサブセクション、課題内容コピペ)、

\documentclass[a4j,11pt]{jarticle}
\usepackage[top=25truemm,  bottom=30truemm,
            left=25truemm, right=25truemm]{geometry}
\title{システムプログラミング1 \\
       レポート}

% ToDo: 自分自身の氏名と学生番号に書き換える
\author{氏名: 今田 将也 (IMADA, Masaya) \\
        学生番号: 09430509}

% ToDo: 教員の指示に従って適切に書き換える
\date{出題日: 2019年xx月xx日 \\
      提出日: 20xx年xx月xx日 \\
      締切日: 20xx年xx月xx日 \\}  % 注:最後の\\は不要に見えるが必要.

\begin{document}
\maketitle

% 目次つきの表紙ページにする場合はコメントを外す
%{\footnotesize \tableofcontents \newpage}

%%%%%%%%%%%%%%%%%%%%%%%%%%%%%%%%%%%%%%%%%%%%%%%%%%%%%%%%%%%%%%%%
\section{概要} \label{chap:abstract}
%%%%%%%%%%%%%%%%%%%%%%%%%%%%%%%%%%%%%%%%%%%%%%%%%%%%%%%%%%%%%%%%

本演習では,プログラミングに関する理解を深めるために不可欠なアセンブラとC言語の境界部分についての演習を行った.具体的には,SPIMというMIPS CPUシミュレータのハードウェア上にC言語とアセンブリ言語を仕様して文字の表示と入力のためのシステムコールライブラリを作成した.さらに,そのライブラリを使用して printf 及び gets 相当を C言語で作成する. 最後に,それらを利用した応用プログラムを動作させた.

なお、与えられた課題内容を以下に述べる.

\subsection{課題内容}\label{kadai}
以下の課題についてレポートをする. プログラムは,MIPSアセンブリ言語で記述し,SPIMを用いて動作を確認している.
\begin{description}
\item[課題1-1] 教科書A.8節 「入力と出力」に示されている方法と, A.9節 最後「システムコール」に示されている方法のそれぞれで "Hello World" を表示せよ.両者の方式を比較し考察せよ.
\item[課題1-2]アセンブリ言語中で使用する .data, .text および .align とは何か解説せよ. 下記コード中の 6行目の .data がない場合,どうなるかについて考察せよ.
\begin{verbatim}
 1:         .text
 2:         .align  2
 3: _print_message:
 4:         la      $a0, msg
 5:         li      $v0, 4
 6:         .data
 7:         .align  2
 8: msg:
 9:         .asciiz "Hello!!\n"
10:         .text
11:         syscall
12:         j       $ra
13: main:
14:         subu    $sp, $sp, 24
15:         sw      $ra, 16($sp)
16:         jal     _print_message
17:         lw      $ra, 16($sp)
18:         addu    $sp, $sp, 24
19:         j       $ra
\end{verbatim}
\item[課題1-3]教科書A.6節 「手続き呼出し規約」に従って,関数 fact を実装せよ. (以降の課題においては,この規約に全て従うこと) fact をC言語で記述した場合は,以下のようになるであろう. 
\begin{verbatim}
 1: main()
 2: {
 3:   print_string("The factorial of 10 is ");
 4:   print_int(fact(10));
 5:   print_string("\n");
 6: }
 7: 
 8: int fact(int n)
 9: {
10:   if (n < 1)
11:     return 1;
12:   else
13:     return n * fact(n - 1);
14: }
\end{verbatim}
\item[課題1-4]素数を最初から100番目まで求めて表示するMIPSのアセンブリ言語プログラムを作成してテストせよ. その際,素数を求めるために下記の2つのルーチンを作成すること.

\end{description}

\section{課題}
 \subsection{課題1ー1}
  \subsubsection{考察}

\section{感想}
\end{document}
