\documentclass[a4j,11pt]{jarticle}
\usepackage[top=25truemm,  bottom=30truemm,
            left=25truemm, right=25truemm]{geometry}
\usepackage{ascmac}
\usepackage{verbatim}
\title{システムプログラミング1 \\
       レポート}

% ToDo: 自分自身の氏名と学生番号に書き換える
\author{氏名: 今田 将也 (IMADA, Masaya) \\
        学生番号: 09430509}

% ToDo: 教員の指示に従って適切に書き換える
\date{出題日: 2019年10月07日 \\
      提出日: 2019年11月20日 \\
      締切日: 2019年11月25日 \\}  % 注:最後の\\は不要に見えるが必要.

\begin{document}
\maketitle

% 目次つきの表紙ページにする場合はコメントを外す
%{\footnotesize \tableofcontents \newpage}

%%%%%%%%%%%%%%%%%%%%%%%%%%%%%%%%%%%%%%%%%%%%%%%%%%%%%%%%%%%%%%%%
\section{概要} \label{chap:abstract}
%%%%%%%%%%%%%%%%%%%%%%%%%%%%%%%%%%%%%%%%%%%%%%%%%%%%%%%%%%%%%%%%

本演習では,プログラミングに関する理解を深めるために不可欠なアセンブラとC言語の境界部分についての演習やMIPSアーキテクチャとアセンブリ言語,アセンブラ特有の記法.また,メモリや入出力,文字と文字列の扱い,レジスタやスタックを用いた手続き呼出の仕組みの演習を行った.具体的には,SPIMというMIPS CPUシミュレータのハードウェア上にC言語とアセンブリ言語を仕様して文字の表示と入力のためのシステムコールライブラリを作成した.さらに,そのライブラリを使用して printf 及び gets 相当を今後C言語で作成する.本課題で実行した結果は,xspimというエミュレータのコンソール上の結果を表示している.

なお、与えられた課題内容を以下に述べる.

\subsection{課題内容}\label{kadai}
以下の課題についてレポートをする. プログラムは,MIPSアセンブリ言語で記述し,SPIMを用いて動作を確認している.
\begin{description}
\item[課題1-1] 教科書A.8節 「入力と出力」に示されている方法と, A.9節 最後「システムコール」に示されている方法のそれぞれで "Hello World" を表示せよ.両者の方式を比較し考察せよ.
\item[課題1-2]アセンブリ言語中で使用する \verb|.data|, \verb|.text| および \verb|.align| とは何か解説せよ. 下記コード中の 6行目の .data がない場合,どうなるかについて考察せよ.
\begin{verbatim}
 1:         .text
 2:         .align  2
 3: _print_message:
 4:         la      $a0, msg
 5:         li      $v0, 4
 6:         .data
 7:         .align  2
 8: msg:
 9:         .asciiz "Hello!!\n"
10:         .text
11:         syscall
12:         j       $ra
13: main:
14:         subu    $sp, $sp, 24
15:         sw      $ra, 16($sp)
16:         jal     _print_message
17:         lw      $ra, 16($sp)
18:         addu    $sp, $sp, 24
19:         j       $ra
\end{verbatim}
\item[課題1-3]教科書A.6節 「手続き呼出し規約」に従って,関数 fact を実装せよ. (以降の課題においては,この規約に全て従うこと) fact をC言語で記述した場合は,以下のようになるであろう. 
\begin{verbatim}
 1: main()
 2: {
 3:   print_string("The factorial of 10 is ");
 4:   print_int(fact(10));
 5:   print_string("\n");
 6: }
 7: 
 8: int fact(int n)
 9: {
10:   if (n < 1)
11:     return 1;
12:   else
13:     return n * fact(n - 1);
14: }
\end{verbatim}
\item[課題1-4]素数を最初から100番目まで求めて表示するMIPSのアセンブリ言語プログラムを作成してテストせよ.その際,素数を求めるために下記の2つのルーチンを作成すること.

  \begin{table}[h]
    \begin{tabular}{|l|l|}
      \hline
      \multicolumn{1}{|c|}{関数名} & \multicolumn{1}{c|}{概要} \\ \hline
      test\_prime(n)            & nが素数なら1,そうでなければ0を返す     \\ \hline
      main()                    & 整数を順々に素数判定し,100個プリント    \\ \hline
    \end{tabular}
  \end{table}
C言語で記述したプログラム例: 
\begin{verbatim}
 1: int test_prime(int n)
 2: {
 3:   int i;
 4:   for (i = 2; i < n; i++){
 5:     if (n % i == 0)
 6:       return 0;
 7:   }
 8:   return 1;
 9: }
10: 
11: int main()
12: {
13:   int match = 0, n = 2;
14:   while (match < 100){
15:     if (test_prime(n) == 1){
16:       print_int(n);
17:       print_string(" ");
18:       match++;
19:     }
20:     n++;
21:   }
22:   print_string("\n");
23: }
\end{verbatim}
実行結果(行を適当に折り返している):
\begin{verbatim}
  2   3   5   7  11  13  17  19  23  29
 31  37  41  43  47  53  59  61  67  71
 73  79  83  89  97 101 103 107 109 113
127 131 137 139 149 151 157 163 167 173
179 181 191 193 197 199 211 223 227 229
233 239 241 251 257 263 269 271 277 281
283 293 307 311 313 317 331 337 347 349
353 359 367 373 379 383 389 397 401 409
419 421 431 433 439 443 449 457 461 463
467 479 487 491 499 503 509 521 523 541
\end{verbatim}

\item[課題1-5]素数を最初から100番目まで求めて表示するMIPSのアセンブリ言語プログラムを作成してテストせよ. ただし,配列に実行結果を保存するように main 部分を改造し, ユーザの入力によって任意の番目の配列要素を表示可能にせよ. 

C言語で記述したプログラム例:
\begin{verbatim}
 1: int primes[100];
 2: int main()
 3: {
 4:   int match = 0, n = 2;
 5:   while (match < 100){
 6:     if (test_prime(n) == 1){
 7:       primes[match++] = n;
 8:     }
 9:     n++;
10:   }
11:   for (;;){
12:     print_string("> ");
13:     print_int(primes[read_int() - 1]);
14:     print_string("\n");
15:   }
16: }
\end{verbatim}
実行例:
\begin{verbatim}
> 15
47
> 100
541
\end{verbatim}
\end{description}

\subsection{xspimの実行方法}
\begin{verbatim}
xspim -mapped_io&
\end{verbatim}
でコンソール上で実行後,必要なアセンブリファイルをloadし,runすることで実行した.
\section{課題レポート}

 \subsection{課題1-1}

  \subsubsection{作成したプログラム}

    A.8節「入力と出力」に示されている方法
    \begin{verbatim}
     1	        .text
     2	        .align 2
     3	main:
     4	        li      $a0,72
     5	putc:
     6	        lw      $t0,0xffff0008
     7	        li      $t1,1
     8	        and     $t0,$t0,$t1
     9	        beqz    $t0,putc
    10	        sw      $a0,0xffff000c
    11	        li      $a0,101
    12	putc2:
    13	        lw      $t0,0xffff0008
    14	        li      $t1,1
    15	        and     $t0,$t0,$t1
    16	        beqz    $t0,putc2
    17	        sw      $a0,0xffff000c
    18	        li      $a0,108
    19	putc3:
    20	        lw      $t0,0xffff0008
    21	        li      $t1,1
    22	        and     $t0,$t0,$t1
    23	        beqz    $t0,putc3
    24	        sw      $a0,0xffff000c
    25	        li      $a0,108
    26	putc4:
    27	        lw      $t0,0xffff0008
    28	        li      $t1,1
    29	        and     $t0,$t0,$t1
    30	        beqz    $t0,putc4
    31	        sw      $a0,0xffff000c
    32	        li      $a0,111
    33	putc5:
    34	        lw      $t0,0xffff0008
    35	        li      $t1,1
    36	        and     $t0,$t0,$t1
    37	        beqz    $t0,putc5
    38	        sw      $a0,0xffff000c
    39	        li      $a0,32
    40	
    41	putc6:
    42	        lw      $t0,0xffff0008
    43	        li      $t1,1
    44	        and     $t0,$t0,$t1
    45	        beqz    $t0,putc6
    46	        sw      $a0,0xffff000c
    47	        li      $a0,87
    48	putc7:
    49	        lw      $t0,0xffff0008
    50	        li      $t1,1
    51	        and     $t0,$t0,$t1
    52	        beqz    $t0,putc7
    53	        sw      $a0,0xffff000c
    54	        li      $a0,111
    55	putc8:
    56	        lw      $t0,0xffff0008
    57	        li      $t1,1
    58	        and     $t0,$t0,$t1
    59	        beqz    $t0,putc8
    60	        sw      $a0,0xffff000c
    61	        li      $a0,114
    62	putc9:
    63	        lw      $t0,0xffff0008
    64	        li      $t1,1
    65	        and     $t0,$t0,$t1
    66	        beqz    $t0,putc9
    67	        sw      $a0,0xffff000c
    68	        li      $a0,108
    69	putc10:
    70	        lw      $t0,0xffff0008
    71	        li      $t1,1
    72	        and     $t0,$t0,$t1
    73	        beqz    $t0,putc10
    74	        sw      $a0,0xffff000c
    75	        li      $a0,100
    76	putc11:
    77	        lw      $t0,0xffff0008
    78	        li      $t1,1
    79	        and     $t0,$t0,$t1
    80	        beqz    $t0,putc11
    81	        sw      $a0,0xffff000c
    82	        j       $ra
    \end{verbatim}

    A.9節「システムコール」に示されている方法k1.s
    \begin{verbatim}
     1	        .data
     2	        .align 2
     3	str:    .asciiz "Hello World"
     4	
     5	        .text
     6	        .align 2
     7	
     8	main:   li  $v0,4   #print_strのシスコールを$v0にロード
     9	        la  $a0,str #プリントする文字列のアドレスをsyscallの引数
    10	                    #$a0にロードアドレス命令を行う
    11	        syscall
    12	        j   $ra     #$raレジスタへ戻り、プログラム終了
      \end{verbatim}

  \subsubsection{考察}
  
  前者での文字の出力は,野蛮な方法である.計算機ごとに変り得るアドレスの\verb|0xffff000c|を意識しつつ使うのは面倒であり,アドレスを知る術がない場合実装するのが不可能である.また,仮に他のプログラムも同時に印刷しようとした場合に競合が発生する可能性もある.このプログラムは印刷が可能になるまで待機してから印刷を行っているが,待たずに印刷するようなプログラムを作成した場合,機器の破壊につながることもあるだろう.

それに比べて,システムコールはカーネルごとに引数の意味が異なって,そのアドレスが変化したとしてもプログラムを変更する必要がなく,他のプログラムとの競合も調整してもらえるため,安全にプログラムを走行することができる.システムコール命令を用いることで安全にユーザプログラムからカーネルやメモリ資源を保護することができ、カーネルに所望の処理を依頼することができる.

 \subsection{課題1-2}
\subsubsection{実行結果}
  \begin{screen}
    課題のコードの実行結果
    \end{screen}
    \begin{verbatim}
Hello!!
    \end{verbatim}
  \begin{screen}
    6行目の.dataをコメントアウトした場合の場合の実行結果
    \end{screen}
    \begin{verbatim}
X\200}B
    \end{verbatim}
  \subsubsection{考察}
  まず,\verb|.data|,\verb|.text|とはメモリ中のどこにデータやテキストを配置するかを制御するためのアセンブラ指令である.本講義で使用した\verb|SPIM|はテキストとデータのセグメントを分割してメモリ中に並べていくようになっている.しかし,テキストとデータは最終的にどちらも数値であるため,どちらをどこに配置するかアセンブラでは決定できず,プログラマ側で指定する必要がある.また,テキストは通常書き換わることはないのでデータと違って読み込み専用のメモリ上に配置することができる.また,異なるプロセスで同じプログラムを実行する場合でもテキストは同一なので共有することも可能になる.このように,データとテキストを意識して区別することで効率的にプロセスを実行できる.

  課題中の6行目の\verb|.data|がない場合,\verb|xspim|で\verb|load|を実行した時点で,\verb|.asciiz|の\verb|"Hello!!\n"|がデータセグメントであるため,テキストセグメントに配置することができないというエラー表示が出る.そして,実行すると\verb|X\200}B|と表示された.もう一度実行すると,\verb|@\207Y^B|と異なった表示がされた.\verb|.data|がなくなったことで,\verb|$a0|レジスタに\verb|msg|の示すアドレスの先に\verb|"Hello!!\n"|ではない内容が存在するようになり,印刷する際にそこの内容を表示していると考えられる.その内容は実行ごとに変わるため,表示結果も変わっていると考える.
 \subsection{課題1-3}
  \subsubsection{実装内容}
  プログラムは大きく分けて,\verb|main|と\verb|fact|部に分かれる.main部では,まず手続き呼出規約に従ってスタックを確保した.そして課題のフローに従って,引数を設定しfact部を実行後,結果をシステムコールにて印刷し各種のアドレスを復元し,スタックポインタをポップし,プログラムを終了する.fact部では,まずmain部同様に手続き呼出規約に従ってスタックを確保し,再帰的に引数を渡すことが出きるように確保した.引数が$0$より大きいなら再帰処理の\verb|factsub|部に\verb|jal|命令を実行,$0$以下なら$1$を返し,一つ前のfactルーチンを呼出し、その計算結果に引数をかけていくという処理を再帰的に繰り返した.factルーチン終了後は戻りアドレスを復元し,スタックポインタをポップする処理を行いルーチンを終了させている.

  \subsubsection{作成したプログラム}
  10の階乗を再帰的に求めるプログラム
\begin{verbatim}
     1	        .data
     2	        .align 2
     3	str:
     4	        .ascii "The factorial of 10 is "
     5	        .text
     6	        .align 2
     7	print_int:
     8	        li      $v0,1
     9	        syscall
    10	        j       $ra
    11	print_str:
    12	        li      $v0,4
    13	        syscall
    14	        j       $ra        
    15	main:
    16	        subu    $sp,$sp,32  #スタックフレームは32バイト長で確保をする
    17	        sw      $ra,20($sp) #戻りアドレスを退避させる
    18	        sw      $fp,16($sp) #古いフレームポインタを退避
    19	        addiu   $fp,$sp,28  #新しくフレームポインタを設定
    20	        #factを呼び出して戻ってから、syscallで$LCとfactの返り値をプリントする
    21	        li      $a0,10      #引数は10
    22	        jal     fact
    23	        move    $t1,$v0     #返り値をt1に退避       
    24	        la      $a0,str     #a0にテンプレ文のアドレスを記入
    25	        jal     print_str        
    26	        move    $a0,$t1     #factの返り値を保存したt1を$a0に収める
    27	        jal     print_int
    28	        #退避してあったレジスタを復元したあと呼出側へ戻る
    29	        lw      $ra,20($sp) #戻りアドレスを復元
    30	        lw      $fp,16($sp) #フレームポインタを復元
    31	        addiu   $sp,$sp,32  #スタックポインタをポップする
    32	        j       $ra
    33	fact:
    34	        subu    $sp,$sp,32  #スタックフレームは32バイト長
    35	        sw      $ra,20($sp) #戻りアドレスを退避させる
    36	        sw      $fp,16($sp) #古いフレームポインタを退避
    37	        addiu   $fp,$sp,28  #新しくフレームポインタを設定
    38	        sw      $a0,0($fp)  #引数を退避させる# 28($sp)にもってきているもの
    39	        #引数>0かどうかを調べる。
    40	        #引数<=0なら1を返す。
    41	        #引数>0ならfactルーチンを呼出(n-1)を計算し、その結果にnをかける
    42	        #上記を再帰的に繰り返す
    43	        lw      $v0,0($fp)  #nをroadさせておく
    44	        bgtz    $v0,factsub #引数が0より大きければ再帰処理に飛ぶ
    45	        li      $v0,1       #0以下なら1
    46	        j       return
    47	factsub:
    48	        lw      $v1,0($fp)  #nをロードする
    49	        subu    $v0,$v1,1   #n-1
    50	        move    $a0,$v0     #a0にn-1に戻る
    51	        jal     fact
    52	        
    53	        lw      $v1,0($fp)
    54	        mul     $v0,$v0,$v1 #n*fact(n-1)
    55	return: #return処理
    56	        lw      $ra,20($sp) #戻りアドレスを復元
    57	        lw      $fp,16($sp) #フレームポインタを復元
    58	        addiu   $sp,$sp,32  #スタックポインタをポップする
    59	        j       $ra
\end{verbatim}

\subsubsection{実行テスト結果}
\begin{verbatim}
$ xspim -mapped_io&
The factorial of 10 is 3628800
\end{verbatim}

\subsubsection{考察}
本関数は,再帰処理を行うことが指定されていたためスタックを用いて,引数の異なるfactルーチンの値をメモリ上に引数が大きいものから配置し,そのルーチンの引数が小さいものから順に掛け合わすことにより再帰的に計算するように実装を行っている.ループによる実装と大きく異なる点は,自分自身をルーチン内で呼び出していることだから,そのルールを守り,ループとは明らかに異なる実装ができていると考える.
\subsection{課題1-4}
\subsubsection{実装内容}
まず,課題指示にあるようにプログラムは\verb|main|部と\verb|test_prime|部の2つに分けて設計した.
main部では,手続き呼び出し規約に基づきスタックポインタを確保した.そして,$100$個処理するために固定値として\verb|$s0|から\verb|$s3|までそれぞれ最大のループ回数,現在のループ回数,チェック用の数値,判定用の数値を設定した.さらに,1行に10個表示されたら改行するために\verb|$s4|には判定用の数値として$10$を設定している.次に,課題のC言語に倣い\verb|while|処理を行っている.ループ回数が$100$を超えてないか確認し,超えてなければtest\_prime部に飛ぶ.その後,素数ならば内容を表示し,チェック数値,ループ回数をそれぞれインクリメントする.違ったら,表示はせずインクリメントさせる.表示の際に1行に$10$個表示されていたら改行させた.

test\_prime部も課題1-4のC言語に倣っている.初期値として最初の素数である$2$を設定.そして,チェック用の数値をループ回数で割っていき素数かそうでないか判定している.素数なら$1$を返し,そうでないなら$0$を返し処理を抜ける.
\subsubsection{作成したプログラム}
素数を100個表示するプログラム
\begin{verbatim}
     1	        .data
     2	        .align 2
     3	space:
     4	        .asciiz " "
     5	enter:
     6	        .asciiz "\n"
     7	        
     8	        .text
     9	        .align 2        
    10	test_prime:
    11	        subu    $sp,$sp,32      #スタックポインター
    12	        sw      $ra,20($sp)     #$ra
    13	        sw      $fp,16($sp)     #フレームポインター
    14	        addiu   $fp,$sp,28      #フレームポインターのセット
    15	        li      $t0,2           # 1は素数ではないから2を初期値にセット
    16	prime_for:
    17	        beq     $t0, $a0, return1       # return 1 もし n が素数なら (i==n)
    18	        bgt     $t0, $a0, prime_exit    # forループを抜ける. もし n > i なら.
    19	        rem     $t1, $a0, $t0           # $t1 = n % i nをiで割ったあまり
    20	        beqz    $t1, prime_exit         # goto Exit_prime if $t1 == 0
    21	        addi    $t0, $t0, 1             # i++
    22	        j       prime_for               # 再びループへ
    23	return1:
    24	        li      $v0,1           #もしnが素数なら1を代入して返す
    25	        lw      $ra,20($sp)
    26	        lw      $fp,16($sp)
    27	        addiu   $sp,$sp,32
    28	        j       $ra             #mainへ戻る
    29	prime_exit:
    30	        li      $v0,0           #もしnが素数でないなら0を代入して返す
    31	        lw      $ra,20($sp)
    32	        lw      $fp,16($sp)
    33	        addiu   $sp,$sp,32
    34	        j       $ra             #ループを抜ける
    35	main:
    36	        subu    $sp,$sp,32      #stackpointer
    37	        sw      $ra,20($sp)     #$ra
    38	        sw      $fp,16($sp)     #flamepointer
    39	        addiu   $fp,$sp,28      #set fp
    40	        li      $s0,100         #最大ループ回数 (match<100)
    41	        li      $s1,0           #現在のループ回数 (match)
    42	        li      $s2,2           #チェック用の数値 n (n=2)
    43	        li      $s3,1           #test_prime(n) == $s3
    44	        li      $s4,10          #10個表示されたら改行するため.print  \n   
    45	while:  
    46	        beq     $s0,$s1,exit    # s1 == 100 ならば(0~99まで),exitに行く
    47	        move    $a0,$s2         # $s2 => $a0に移動
    48	        jal     test_prime        
    49	        bne     $v0,$s3,else    # $v0 != 1 test_primeから帰ってきた数値で検証する       
    50	        move    $a0,$s2         # 印刷のために,数字を入れる
    51	        li      $v0,1           #1はint
    52	        syscall        
    53	        la      $a0, space      # 空白を印刷
    54	        li      $v0,4           #4は文字列
    55	        syscall        
    56	        addiu   $s1,$s1,1               #現在のループ回数を増加       
    57	        rem     $t2, $s1, $s4           # $t2 = n % i____10個表示されたかどうか
    58	        beqz    $t2, print_enter        # 改行表示 もし $t2(個数) == 0        
    59	else:   
    60	        addiu   $s2, $s2, 1             # n = n + 1 
    61	        j       while                   # whileループを繰り返す
    62	exit:   
    63	        lw      $ra, 20($sp)    # Restore return address
    64	        lw      $fp, 16($sp)    # Restore frame pointer
    65	        addiu   $sp, $sp, 32    # Pop stack frame
    66	        j       $ra             # End this program        
    67	print_enter:
    68	        subu    $sp,$sp,32      #stackpointer
    69	        sw      $ra,20($sp)     #$ra
    70	        sw      $fp,16($sp)     #flamepointer
    71	        addiu   $fp,$sp,28      #set fp
    72	        la      $a0,enter
    73	        li      $v0,4
    74	        syscall
    75	        lw      $ra, 20($sp)    # Restore return address
    76	        lw      $fp, 16($sp)    # Restore frame pointer
    77	        addiu   $sp, $sp, 32    # Pop stack frame
    78	        j       $ra             #return
\end{verbatim}
   \subsubsection{実行テスト結果}
   \begin{verbatim}
$ xspim -mapped_io&
2 3 5 7 11 13 17 19 23 29 
31 37 41 43 47 53 59 61 67 71 
73 79 83 89 97 101 103 107 109 113 
127 131 137 139 149 151 157 163 167 173 
179 181 191 193 197 199 211 223 227 229 
233 239 241 251 257 263 269 271 277 281 
283 293 307 311 313 317 331 337 347 349 
353 359 367 373 379 383 389 397 401 409 
419 421 431 433 439 443 449 457 461 463 
467 479 487 491 499 503 509 521 523 541 
   \end{verbatim}

\subsubsection{考察}
今回の関数は,\verb|$s|とつくレジスタに値や固定値を登録して作成している.プログラム中でこの値が呼び出された先で破壊されることはないため不具合なく実行することができている.しかし,仮に破壊的に利用される場合は,予期しないことが起こる可能性があるため,呼び出される前に一度値を保存して呼び出し後に復元する実装を行うべきだと考える.もしくは,破壊的に利用してもよいレジスタを用いることが望ましいと感じたため,改良の余地がある.
 \subsection{課題1-5}
  \subsubsection{実装内容}
  課題1-4の\verb|main|部と\verb|test_prime|部を用いている.また,\verb|.space|で$400$バイト分の配列の確保も行った.課題1-4と大きく異なるのは,素数を画面に表示するのではなく配列に格納するように実装しているところである.格納元レジスタを倍々にすることでアドレスを指定し4バイト単位で格納することで$100$個分の配列機能を実現し,中身の要素を表示している.また,分岐命令を用いて,システムコールにより受け取った値に応じて処理を分岐させた.0ならばプロセスを終了し,負の値もしくは$100$より大きい値ならばエラー文を表示している.
  \subsubsection{作成したプログラム}
   配列で入力に応じて素数を表示するプログラム
\begin{verbatim}
     1	array:
     2	        .space 400      #400バイト分(100個分)の配列用意
     3	        
     4	        .data
     5	        .align 2
     6	space:
     7	        .asciiz " "
     8	enter:
     9	        .asciiz "\n"
    10	start:      
    11	        .asciiz "To quit, type 0\n\n"        
    12	mark:
    13	        .asciiz "\n> "
    14	owari:
    15	        .asciiz "\nGood bye :)\n\n"
    16	excep:
    17	        .asciiz "\nPlease type correct number\n"
    18	        
    19	        .text
    20	        .align 2
    21	        
    22	test_prime:
    23	        subu    $sp,$sp,32      #スタックポインター
    24	        sw      $ra,20($sp)     #$ra
    25	        sw      $fp,16($sp)     #フレームポインター
    26	        addiu   $fp,$sp,28      #フレームポインターのセット
    27	        li      $t0,2           # 1は素数ではないから2を初期値にセット
    28	prime_for:
    29	        beq     $t0, $a0, return1       # return 1 もし n が素数なら (i==n)
    30	        bgt     $t0, $a0, prime_exit    # forループを抜ける. もし n > i なら.
    31	        rem     $t1, $a0, $t0           # $t1 = n % i nをiで割ったあまり
    32	        beqz    $t1, prime_exit         # goto Exit_prime if $t1 == 0
    33	        addi    $t0, $t0, 1             # i++
    34	        j       prime_for               # 再びループへ
    35	return1:
    36	        li      $v0,1           #もしnが素数なら1を代入して返す
    37	        lw      $ra,20($sp)
    38	        lw      $fp,16($sp)
    39	        addiu   $sp,$sp,32
    40	        j       $ra             #mainへ戻る
    41	prime_exit:
    42	        li      $v0,0           #もしnが素数でないなら0を代入して返す
    43	        lw      $ra,20($sp)
    44	        lw      $fp,16($sp)
    45	        addiu   $sp,$sp,32
    46	        j       $ra             #ループを抜ける
    47	main:
    48	        subu    $sp,$sp,32      # stackpointer
    49	        sw      $ra,20($sp)     # $ra
    50	        sw      $fp,16($sp)     # flamepointer
    51	        addiu   $fp,$sp,28      # set fp
    52	        li      $v0, 4          # syscall of print_string
    53	        la      $a0, start      # startラベルの内容を入れる
    54	        syscall                 # 印刷内容 "To quit, type 0"
    55	        li      $s0,100         #最大ループ回数 (match<100)
    56	        li      $s1,0           #現在のループ回数 (match)
    57	        li      $s2,2           #チェック用の数値 n (n=2)
    58	        li      $s3,1           #test_prime(n) == $s3
    59	        la      $a1,array       #$a1にarrayのアドレスを入れる
    60	while:  
    61	        beq     $s0,$s1,exit    # s1 == 100 ならば(0~99まで),exitに行く
    62	        move    $a0,$s2         # $s2 => $a0に移動
    63	        jal     test_prime       
    64	        bne     $v0,$s3,else    # $v0 != 1 test_primeから帰ってきた数値で検証する
    65	        li      $t4, 4          # For array を増加(4バイト単位)
    66	        addu    $a1, $a1, $t4   # $a1 = $a1 + 4
    67	        sw      $s2, 0($a1)     # $s2(素数)=>$a1が指すアドレスの中の先頭にいれる   
    68	        addiu   $s1,$s1,1       # 現在のループ回数を増加      
    69	else:   
    70	        addiu   $s2, $s2, 1     # n = n + 1 
    71	        j       while           # go to Loop
    72	exit:   
    73	        li      $v0, 4          # syscall of print_string
    74	        la      $a0, mark       # 印字内容 ">"
    75	        syscall                 # 印刷
    76	        la      $a1, array      # Initialize $a1
    77	        li      $v0, 5          # For syscall of read_int
    78	        syscall                 # 何番目の素数かを入力する       
    79	        beqz    $v0,end         # 0か文字なら終了
    80	        bltz    $v0,error       # 負ならエラー
    81	        bgtu    $v0,$s0,error   # 100より大きくてもエラー    
    82	        move    $t3, $v0        # 入力された値$v0を $t3 に
    83	        addu    $t3, $t3, $t3   # $t3 = $t3 * 2
    84	        addu    $t3, $t3, $t3   # $t3 = $t3 * 2 4バイト分になる
    85	        addu    $a1, $a1, $t3   # $a1 = $a1       
    86	        lw      $a0, 0($a1)     # $a1 のアドレスから 4バイト(word)取り出して $a0 に代入(load)
    87	        li      $v0, 1          # For syscall of print_int
    88	        syscall                 # Print prime       
    89	        j       exit
    90	error:
    91	        li    $v0, 4            # for syscall of print_string
    92	        la    $a0, excep        # Print error message
    93	        syscall                 # print
    94	        j     exit 
    95	end:
    96	        li      $v0, 4          # for syscall of print_string
    97	        la      $a0, owari      # good,bye
    98	        syscall                 # print       
    99	        lw      $ra, 20($sp)    # Restore return address
   100	        lw      $fp, 16($sp)    # Restore frame pointer
   101	        addiu   $sp, $sp, 32    # Pop stack frame       
   102	        j       $ra             # End this program
\end{verbatim}
   \subsubsection{実行テスト結果}
   \begin{verbatim}
$ xspim -mapped_io&
To quit, type 0
> 100
541
> 14
43
> -1000
Please type correct number
> 199
Please type correct number
> 0
Good bye :)
   \end{verbatim}
\subsubsection{考察}
本プログラムでは,キーボードから数値が入力されることを前提としている.そのため,数値以外の文字列を入力すると何も表示されることなくSPIMが終了する.そして,誤って入力した際に,\verb|BS|キーにて文字や数値を消去することができない.これはユーザの利便性を落とす可能性があるため,入力されたデータをチェックする機構が必要だと考える.

\section{感想}
本演習でのMIPSを用いたプログラミングを通し,オペレーションシステムがメモリにテキストとデータをどう配置していたのかということを垣間見ることができた.また,OSの講義で学んだ概念を実践的に知ることができたので,他の講義の理解を深めることにも繋がった.高級言語と違い,上から順番に実行されていくという性質に慣れるまでに時間がかかったため,特に課題1-3における回帰処理の際に自力では解決することができず解答に頼ってしまったが講義資料や教科書を再三読み直すことで理解ができるようになった.また,スタックポインタやフレームポインタなどの複雑な理屈を深く理解をするまでには至らなかったが,自分が意図するようにレジスタに戻るべき処理を伝えるようにプログラムを記述するまではできた.システムプログラミング2では,今回学んだことをさらに理解してから応用できるようにしたい.

\end{document}
